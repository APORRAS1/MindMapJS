\section{Objetivos}

El principal objetivo de este proyecto es la creación de un editor de mapas mentales online. El editor, frontend, debe ejecutarse completamente en el cliente. Para ello, vamos a utilizar como lienzo de dibujos el canvas de HTML5 y Javascript como lenguaje de desarrollo. 

El usuario podrá navegar por el diagrama con los cursores partiendo desde la idea central. Interactuará con el diagrama de forma que, dependiendo del nodo en el que se encuentre y la acción que realice podrá insertar, modificar, anotar, plegar, etc...

Esta fuerte interacción, provoca que dentro de los objetivos del proyecto, se encuentre la elaboración de una  extensa librería JavaScript, bien estructurada y testeada. 

En todo momento, y en pos de una aplicación lo más estándar posible, se seguirá las especificaciones de la World Wide Web Consortium\footnote{Web oficial de la W3C http://www.w3.org/} (W3C) y la especificación EmacScript.

Como objetivo principal está pues, la universalidad, independencia de sistemas y la inmediatez de uso, sin instalación, siempre actualizada, e incluso la posibilidad de uso en forma local con cualquier navegador actual que sigue el estándar HTML5.  Entre las posibles plataformas de uso se tratará de incluir las plataformas táctiles, especialmente los tablets.
